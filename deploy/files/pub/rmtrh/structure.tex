\documentclass[12pt]{article}

%% General
\usepackage{enumitem}
\usepackage{verbatim} 
\usepackage[utf8]{inputenc}

% URL
\usepackage{hyperref}

% Colour

\usepackage{xcolor}

\begin{comment}
%Header

\usepackage{fancyhdr}
\pagestyle{fancyplain}
\fancyhead{} % No page header - if you want one, create it in the same way as the footers below
\fancyfoot[L]{} % Empty left footer
\fancyfoot[C]{} % Empty center footer
\renewcommand{\headrulewidth}{0pt} % Remove header underlines
\renewcommand{\footrulewidth}{0pt} % Remove footer underlines
\setlength{\headheight}{0pt} % Customize the height of the header
\end{comment}


% Captions
\usepackage[margin=10pt,font=small,labelfont=bf, labelsep=endash]{caption}

%% Geometry

\usepackage[margin=1in]{geometry}


%% Maths

\usepackage[centertags]{amsmath}
\usepackage{amsfonts}
\usepackage{amssymb}
\usepackage{amsthm}
\usepackage{newlfont}

\theoremstyle{plain}
%\theoremstyle{margin}
{\swapnumbers
\newtheorem{thm}{Theorem}[section]
\newtheorem{cor}[thm]{Corollary}
\newtheorem{lem}[thm]{Lemma}
\newtheorem{prop}[thm]{Proposition}
}
%
\theoremstyle{definition}
%\newtheorem{defn}{Definition}[section]
%\newtheorem{nota}{Notation}[section]
%\theoremstyle{margin}
{\swapnumbers
\newtheorem{defn}[thm]{Definition}
\newtheorem{nota}[thm]{Notation}
\newtheorem{parag}[thm]{}
\newtheorem*{parstr}{\addtocounter{thm}{1}\thesection.\arabic{thm}*}
\newtheorem*{intropar}{\addtocounter{thm}{1}\arabic{thm}}
}
%
\theoremstyle{remark} {\swapnumbers
\newtheorem{rem}[thm]{Remark}
\newtheorem{exam}[thm]{Example}
\newtheorem{lemma}[thm]{Lemma}
\newtheorem{claim}[thm]{Claim}
}


%% Graphics
\usepackage{graphicx} 
\graphicspath{{Figures/}} 
\usepackage{subfig}

%% Tables
\usepackage{booktabs}
\usepackage{rotating}

% Allows shading of table cells
\usepackage{colortbl}

% Define a simple command to use at the start of a table row to make it have a shaded background
\newcommand{\gray}{\rowcolor[gray]{.9}}

%% Commands
\newcommand{\bd}{\textbf}
\newcommand{\itt}{\textit}
%
\newcommand{\al}{\alpha}
\newcommand{\be}{\beta}
\newcommand{\de}{\delta}
%
\newcommand{\To}{\longrightarrow}
\newcommand{\RE}{\operatorname{Re}}
\newcommand{\IM}{\operatorname{Im}}
%
\newcommand{\QQ}{\mathbb{Q}}     
\newcommand{\RR}{\mathbb{R}}      
\newcommand{\ZZ}{\mathbb{Z}}      
\newcommand{\CC}{\mathbb{C}}      

\numberwithin{equation}{section}

%% Title

\begin{comment}
\setlength{\droptitle}{-8em} 
\title{	
\normalfont \normalsize \bf
\vspace{-1 em}} 
\end{comment}

%% Referencing

\usepackage[style=nature]{biblatex}
